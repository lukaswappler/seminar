\documentclass[hyperref={pdfpagelabels=false}]{beamer}





%\usepackage{lmodern}

\title{Thema 2.3: Apache Spark}   
\subtitle{Skalierbare verteilte Datenanalyse}
\author{Lukas Wappler} 
\date{\today} 

\usepackage{beamerthemeshadow}
%  \beamersetuncovermixins{\opaqueness<1>{25}}{\opaqueness<2->{15}}
%  sorgt dafuer das die Elemente die erst noch (zukuenftig) kommen 
%  nur schwach angedeutet erscheinen 
\beamersetuncovermixins{\opaqueness<1>{25}}{\opaqueness<2->{15}}
% klappt auch bei Tabellen, wenn teTeX verwendet wird\ldots



%NO FOOTLINE
%gets rid of bottom navigation bars
%\setbeamertemplate{footline}[frame number]{}

%gets rid of bottom navigation symbols
\setbeamertemplate{navigation symbols}{}

%gets rid of footer
%will override 'frame number' instruction above
%comment out to revert to previous/default definitions
\setbeamertemplate{footline}{}


%\setbeamertemplate{frametitle}{\nointerlineskip  
 %   \begin{beamercolorbox}[wd=\paperwidth,ht=2.75ex,dp=1.375ex]{frametitle}
  %      \hspace*{2ex}\insertframetitle \hfill {\insertframenumber} \hspace*{1ex}%
   % \end{beamercolorbox}}

%\addtobeamertemplate{headline}{}{\rule{\paperwidth}{3pt}}

\addtobeamertemplate{headline} 
{
  \leavevmode%
  \hbox{%
  \begin{beamercolorbox}[wd=.333333\paperwidth,ht=2.25ex,dp=1ex,center]{author in head/foot}%
    Lukas Wappler
		%\usebeamerfont{author in head/foot}\insertsection
  \end{beamercolorbox}%
  \begin{beamercolorbox}[wd=.333333\paperwidth,ht=2.25ex,dp=1ex,center]{title in head/foot}%
   
		Thema 2.3: Apache Spark		
  \end{beamercolorbox}%
  \begin{beamercolorbox}[wd=.333333\paperwidth,ht=2.25ex,dp=1ex,right]{date in head/foot}%
    \usebeamerfont{date in head/foot}\insertshortdate{}\hspace*{2em}
    \insertframenumber{} / \inserttotalframenumber \hspace*{2ex} 
  \end{beamercolorbox}}%
  \vskip0pt%
}


\begin{document}




\begin{frame}[plain,noframenumbering]
\titlepage
\end{frame} 


\begin{frame}
\frametitle{Inhaltsverzeichnis}
\setcounter{tocdepth}{1}
\tableofcontents
\end{frame} 



\section{Einleitung} 
\begin{frame}
\frametitle{Einleitung} 
\end{frame}

\section{Apache Spark} 
\begin{frame}
\frametitle{Einleitung} 
\end{frame}

\subsection{Kern-Bibliotheken / Komponenten}
\begin{frame} 

\end{frame}

\subsubsection{Spark-Core}
\begin{frame} 

\end{frame}

\subsubsection{RDD’s}
\begin{frame} 

\end{frame}


\subsection{SQL-Abfragen}
\subsubsection{Spark-SQL}
\begin{frame} 

\end{frame}


\subsubsection{Data Frames}
\begin{frame} 

\end{frame}

\subsection{  Verarbeitung von Datenströmen (Spark-Streaming)}
\begin{frame} 

\end{frame}


\subsection{  Berechnungen auf Graphen (GraphX)}
\begin{frame} 

\end{frame}

\subsection{  Maschinelles Lernen (MLlib)}
\begin{frame} 

\end{frame}

\subsection{  Skalierung von R Programmen (SparkR)}
\begin{frame} 

\end{frame}



\section{Mehrere Komponenten im Verbund}
\begin{frame} 

\end{frame}

\section{  Performance}
\begin{frame} 

\end{frame}

 \subsection{Besonderheiten bei der Speichernutzung}
\begin{frame} 

\end{frame}

 \subsection{Netzwerk und I/O-Traffic}
\begin{frame} 

\end{frame}


\section{Nutzung und Verbreitung}
\begin{frame} 

\end{frame}


\section{ Fazit}
\begin{frame} 

\end{frame}

\section{Ausblick und Weiterentwicklung}
\begin{frame} 

\end{frame}



\end{document}
\newpage
\chapter{Ausblick \& Weiterentwicklung} 

Immer mehr Firmen führen Apache Spark ein oder nutzen es bereits. Dieser Trend sollte auch weiterhin so bleiben.
\\ \\
\noindent
Seit der Einführung von Apache Spark im Jahr 2010 wird die Software kontinuierlich verbessert und weiterentwickelt. 
Vieles hat die Community dazu beigetragen, die aufgrund der Open-Source Software dazu in der Lage ist aktiv daran mit zu arbeiten. 
Die Kommunikation innerhalb der Community findet im wesentlichen über offizielle Mailinglisten und einem Ticket-System der Apache Foundation statt.
Der Code liegt auf GitHub\footnote{\textbf{GitHub} ist ein webbasierter Onlinedienst, der die Möglichkeit bietet Softwareprojekte mit der Versionsverwaltung Git zu verwalten.} und ist öffentlich für jeden zugänglich. 
Bis zum 10.04.2017 gab es bereits 51 Releases oder Release-Kandidaten, 19,365 commits und 1,053 contributors\footnote{\textbf{contributors}: Sind Personen, die zum Projekt mit Schreiben von Code beigetragen haben.}.
Auch das wird zukünftig weiter gehen. Im ersten Quartal 2017 gab es über 717 commits. Ein Einbruch der Aktivität ist momentan nicht zu erkennen.\footnote{Vgl. \cite{GITHUB}}


\noindent
%https://databricks.com/blog/2016/05/11/apache-spark-2-0-technical-preview-easier-faster-and-smarter.html
Von der Version 1.6 auf die Version 2.0 gab es nochmal eine relativ starke Performancesteigerung. Vermutlich wird man solche Performancesteigerungen nicht mehr so leicht erreichen. Trotzdem sollten Geschwindigkeiten bei der Verarbeitung solcher großen Datenmengen auch zukünftig noch etwas nach unten verändern. Eine Übersicht der Performanceänderungen ist in der Tabelle \ref{tab:spark_2_0} zu sehen.\footnote{Vgl. \cite{DATABRICK_SPARK_2_0}}


\begin{table}[h]
  \centering
		
		  \begin{tabular}[t]{|l|l|l|}
    \hline
		
		\rowcolor[gray]{.9}
		
				primitive	 & Spark 1.6 &  Spark 2.0 \\ \hline				
				filter &	15ns &	1.1ns \\ \hline				
				sum w/o group &	14ns &	0.9ns \\ \hline				
				sum w/ group &	79ns &	10.7ns \\ \hline				
				hash join	& 115ns	& 4.0ns \\ \hline				
				sort (8-bit entropy)	& 620ns	 & 5.3ns \\ \hline				

  \end{tabular}
		
  \caption{Kosten pro Zeile (cost per row) auf einem einzelnen Thread}\label{tab:spark_2_0}
\end{table}



\noindent
Zukünftig ist denkbar, das noch weitere Komponenten so wie zum Beispiel SparkR dazu kommen. Auch das Anbinden weiterer Datenquellen wird sehr wahrscheinlich weiter vorangetrieben werden.